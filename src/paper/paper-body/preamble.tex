% !TEX root = ../research_pres.tex

\usepackage[english]{babel}
\usepackage{a4wide}
\usepackage[utf8]{inputenc}
\usepackage{csquotes}
\usepackage{float, afterpage, rotating, graphicx}
\usepackage{epstopdf}
\usepackage{longtable, booktabs, tabularx}
\usepackage{fancyvrb, moreverb, relsize}
\usepackage{eurosym, calc, chngcntr}
\usepackage{amsmath, amssymb, amsfonts, amsthm, bm}
\usepackage{caption}
\usepackage{mdwlist}
\usepackage{xfrac}
\usepackage{setspace}
\usepackage{xcolor}
\usepackage{enumerate}
\usepackage{geometry}
% \usepackage{pdf14} % Enable for Manuscriptcentral -- can't handle pdf 1.5
% \usepackage{endfloat} % Enable to move tables / figures to the end. Useful for some submissions.

\usepackage{subfig}
\usepackage{tgpagella}
\usepackage{unicode-math}
% \setmainfont[
%   BoldFont=texgyrepagella-bold.otf,
%   ItalicFont=texgyrepagella-italic.otf,
%   BoldItalicFont=texgyrepagella-bolditalic.otf]
%   {texgyrepagella-regular.otf}
% \setmathfont{texgyrepagella-math.otf}

% \setmainfont{LibreBaskerville}[%
%   Path = /home/ignaszak/.local/share/fonts/ ,
%   UprightFont = *-Regular ,
%   BoldFont = *-Bold ,
%   ItalicFont = *-Italic ,
%   Extension = .ttf
% ]
% \setmainfont[
%   BoldFont= /home/ignaszak/.local/share/fonts/LibreBaskerville-Bold.ttf,
%   ItalicFont= /home/ignaszak/.local/share/fonts/LibreBaskerville-Italic.ttf]
  % {/home/ignaszak/.local/share/fonts/LibreBaskerville-Regular.ttf}
% \setmathfont{texgyrepagella-math.otf}


\linespread{1.05}         % Palladio needs more leading (space between lines)
\usepackage{epstopdf}

\usepackage[
    natbib=true,
    bibencoding=inputenc,
    bibstyle=authoryear,
    citestyle=authoryear-comp,
    maxcitenames=3,
    maxbibnames=10,
    useprefix=false,
    sortcites=true,
    backend=biber,
    doi=false, isbn=false, url=false, eprint=false
]{biblatex}

\AtBeginDocument{\toggletrue{blx@useprefix}}
\AtBeginBibliography{\togglefalse{blx@useprefix}}
\setlength{\bibitemsep}{1.5ex}

\usepackage[unicode=true]{hyperref}
\hypersetup{
    colorlinks=true,
    linkcolor=black,
    anchorcolor=black,
    citecolor=black,
    filecolor=black,
    menucolor=black,
    runcolor=black,
    urlcolor=black
}


\widowpenalty=10000
\clubpenalty=10000

\setlength{\parskip}{1ex}
%\setlength{\parindent}{.5ex}
\setstretch{1.5}

\theoremstyle{plain}

\newtheorem{assumpt}{Assumption}
\newtheorem{thrm}[assumpt]{Theorem}
\newtheorem{propos}[assumpt]{Proposition}
%\theoremstyle{remark}
\newtheorem{lemma}[assumpt]{Lemma}
\theoremstyle{definition}
\newtheorem{defin}[assumpt]{Definition}
%declaration of new variables
\newcommand{\ud}{\,\!\mathrm{d}}
\newcommand{\dt}{\! \! \ud t}
\newcommand{\di}{ \! \! \ud i}

\renewcommand{\iff}{\ensuremath{\Leftrightarrow}}
\newcommand{\into}{\ensuremath{\rightarrow}}
\newcommand{\lub}{\ensuremath{\,\vee\,}}
\newcommand{\oraz}{\ensuremath{\,\wedge\,}}
\newcommand{\then}{\ensuremath{\,\Rightarrow \,}}
\newcommand{\neht}{\ensuremath{\Leftarrow}}

\newcommand{\tinf}{_{t=0}^{\infty}}

\newcommand{\R}{\ensuremath{\mathbb{R}}} %Real numbers

%new mathematical operators
\DeclareMathOperator{\Lagr}{\mathcal{L}}
\DeclareMathOperator{\Ham}{\mathcal{H}}
\DeclareMathOperator*{\argmin}{\text{arg min}}
\DeclareMathOperator*{\argmax}{\text{arg max}}
\DeclareMathOperator{\E}{\mathbb{E}} % expected value

%new environments
\newcommand{\przypis}[1]{\footnotesize{#1}} %przypis{blahblah}, small font
\newcommand{\myvector}[3]{\begin{bmatrix} {#1} \\ {#2} \\ \vdots \\ {#3} \end{bmatrix}}

%visual customization
\renewcommand{\labelitemi}{$\bullet$}
\renewcommand{\textapprox}{\raisebox{0.5ex}{\texttildelow}}
